% -*- root: main.tex -*-
\anonsection{Введение}

В настоящее время наибольшую угрозу информационной безопасности несут вредоносные программы. Некоторые разновидности вредоносных программ используют алгоритмы генерирования доменных имен для определения адресов управляющих серверов. Подобные алгоритмы позволяют защитить управляющие вредоносные сервера от однократного отключения или добавления адресов в черные списки. Чаще всего данные алгоритмы используются в крупных ботнетах. 

Ботнет - некоторая сеть, в том числе компьютерная, состоящая из устройств - ботов, со специально запущенным вредоносным программных обеспечением. Чаще всего боты инфицируются посредством вредоносного программного обеспечения, полученного из сети Интернет. Однако, путём инфицирования может служить также локальная сеть или устройства ввода, например, флэш накопители. Ботнеты являются наиболее распространенным средством кибер атак. Они управляются его создателем при помощи специальных управляющих командных серверов (Command and Control Servers). Большинство из крупных ботнетов используются для монетизации различными способами, такими как: распределенные атаки отказа в обслуживании (DDoS атаки), продажа Drive By Download, атаки на клиентов дистанционного банковского обслуживания, для спама и проведения фишинговых атак. Эти и другие угрозы, которые несут в себе ботсети, освещены в статье [10].

Для удержания контроля над ботами и их управления ботнеты используют множество способов. Это могут быть одноранговые сети, почтовые протоколы, социальные сети или анонимные сети, такие как TOR или i2p.
Однако, самым распространенным на данный момент методом являются алгоритмы генерации доменных имен (Domain Generation Algotithms).

Они позволяют удерживать контроль над управляющими серверами. В основном подобные алгоритмы используются в крупных ботсетях. Например, одним из первых случаев их использования был компьютерный червь Conficker в 2008 году. На сегодняшний день подобных вредоносных программ насчитываются десятки, каждая из которых представляет серьезную угрозу. Помимо этого, алгоритмы совершенствуются, их обнаружение становится сложнее. Например, осенью 2014 года была обнаружена новая версия ботнета Matsnu, в которой для генерации доменов используются существительные и глаголы из встроенного списка. Это позволяет семейству этих вредоносных программ обходить существующие механизмы распознования вредоносных доменных имен. Подробнее история развития алгоритмов генерации доменных имен рассмотрена в статье [6].

Целью данной работы является разработка модели на основе методов машинного обучения для распознавания и классификации вредоносных доменных имен, полученных при помощи анализа алгоритмов генерации доменных имен, а также получение количественной оценки качества классификации на сформированной выборке доменных имен.

Для достижения поставленной цели были поставлены три задачи. Во-первых, подготовка эксперементальных данных. При этом необходимо составить выборки не только для легетимных и вредоносных имен, но и выборки по отдельным классам вредоносных программ. Под вредоносным доменном именем в данной работе понимается доменное имя, искуственно полученное в результате работы алгоритма генерации доменных имен. Вторая задач - поиск и реализация существующих подходов, их сравнительный анализ на подготовленных данных. Третья - разработка и апробация собственной модели на основе реккурентных нейронных сетей.

Новизна данной работы заключается в решении третьей задачи. Так, для классификации предлагается использование модели, построенной на основе рекуррентной нейронной сети, что является новым направлением исследования для решения проблемы обнаружения и распознавания вредоносных доменных имен. Данный подход позволяет улучшить качество классификации по сравнению с существующими подходами решения данного класса задач. Поэтому, как считает автор, рассмотрение нейросетей для предотвращения современных угроз, связанных с ботнетами и алгоритмами генерации доменных имен является актуальной проблемой информационной безопасности не только в настоящее время, но и в будущем.

В главе \ref{dga} рассматриваются принципы работы алгоритмов генерации доменных имен. Приводится простейший пример реализации, а также особенности каждого из рассмотренных таких алгоритмов.

В главе \ref{be_class} производится обзор существующей литературы, затрагивающей данную проблематику. Выделяются наиболее эффективные модели и подходы к решению задачи распознавания вредоносных доменных имен.

В главе \ref{ner_network} рассмотрены теоретические основы и архитектура разработанной модели. Затрагиваются вопросы, связанные с нейронными сетями, рекуррентными нейронными сетями, моделью Long short-term memory.

Глава \ref{experiment} посвящена практической реализации существующих подходов - раздел \ref{be_class_exp}.В этом разделе обсуждаются вопросы выбора параметров для классификаторов и полученные результаты. Раздел \ref{lstm_class_exp} посвящен разработанной модели, получению количественной оценки качества классификации на сформированной выборке доменных имен
\clearpage