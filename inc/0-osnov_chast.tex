\section{Алгоритмы Генерации Доменных Имен}
Алгоритмы Генерации Доменных Имен (DGA) представляют собой алгоритмы, используемые вредоносным программным обеспечением (malware) для генерации большого количества псевдослучайных доменных имен, которые позволят им установить соединение с управляющим командным центром.

    \subsection{Общие принципы работы}
    \subsection{Расмотренные алгоритмы}
    В ходе работы были проанализированы 8 разновидностей алгоритмов генерации доменных имен, а именно:
        \item{Cryptolocker}
        \item{Zeus}
        \item{PushDo}
        \item{Rovnix}
        \item{Tinba}
        \item{Conficker}
        \item{Matsnu}
        \item{Ramdo}

\section{Существующие подходы к классификации}
    \subsection{Naive Bayes}
    \subsection{Logistic Regression}
    \subsection{Random Forest}
    \subsection{Extra Tree Forest}
    \subsection{Voting Classification}
\clearpage

\section{Нейронные сети}
    \subsection{Реккурентные нейронные сети}
    \subsection{LSTM}
    \subsection{Biderectial LSTM}
    \subsection{Механизм внимания}
\clearpage

\section{Эксперимент}
    \subsection{Существующие подходы}
    \subsection{Результаты LSTM}
    \subsection{Сравнительный анализ}
\clearpage