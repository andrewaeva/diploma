\section{Ботнеты}
Ботнеты - некоторая сеть, в том числе компьютерная, состоящая из устройств (ботов), со специально запущенным вредоносным программных обеспечением. Чаще всего боты инфицируются посредством вредоносного программного обеспечения, полученного из сети Интернет. Однако путём инфицирования может служить также локальная сеть или устройства ввода, например флэш накопители. Ботнеты являются наиболее распространенными средствами кибер атак. Они управляются его создателем при помощи специальных управляющих командных серверов (Command and Control Servers). Большинство из них используются для монетизации различными способами, такими как: распределенные атаки отказ в обслуживании (DDoS атаки), продажа Drive By Download, атаки на клиентов дистанционного банковского обслуживания, для спама и проведения фишинговых атак.

Для удержания контроля над ботами и их управления Ботнеты используют множество способов. Это может быть p2p сети, всевозможные mail протоколы, ....
Однако самым распространенным на данный момент является Алгоритмы Генерации Доменных Имен (Domain Generation Algotithms)
\clearpage

\section{Алгоритмы Генерации Доменных Имен}
Алгоритмы Генерации Доменных Имен (DGA) представляют собой алгоритмы, используемые вредоносным программным обеспечением (malware) для генерации большого количества псевдослучайных доменных имен, которые позволят им установить соединение с управляющим командным центром.
\subsection{Cryptolocker}
cryptolocker
\subsection{Zeus}
\subsection{PushDo}
\subsection{Rovnix}
\subsection{Tinba}
\subsection{Conficker}
\subsection{Matsnu}
\subsection{Ramdo}

\section{Алгоритмы машинного обучения}
\subsection{Naive Bayes}
\subsection{Logistic Regression}
\subsection{Random Forest}
\subsection*{Extra Tree Forest}
\subsection*{Voting Classification}
\clearpage

\section{Реккурентные нейронные сети}
\subsection{LSTM}
\clearpage

\section{Эксперимент}
\subsection{Обучающая выборка}
\subsection{Результаты алгоритмов}
\subsection{Результаты LSTM}
\subsection{Сравнительный анализ}


\clearpage