\section{Алгоритмы Генерации Доменных Имен}
Алгоритмы Генерации Доменных Имен (DGA) представляют собой алгоритмы, используемые вредоносным программным обеспечением (malware) для генерации большого количества псевдослучайных доменных имен, которые позволят им установить соединение с управляющим командным центром. Рассмотрим общие принципы работы таких алгоритмов.
    \subsection{Общие принципы работы}\label{work_princip}
        Общий принцип работы представлен на рис \ref{dga_work}. В общем случае вредосному файлы необходим какой-либо параметр для инициализации Генератора Пседослучайных Чисел (ГСПЧ). В качестве этого параметра может выступать любой параметр, который будет известен вредоносному файлу и владельцу ботнета. В нашем случае - это значение текущей даты и времени. Вредоносный файл, используя протокол HTTP посылает запрос на сайт cnn.com. В ответ на этот запрос cnn.com возращает в заголовках HTTP ответа текущие время и дату в формате GMT. Владелец ботнета таким же способом получает текущее время и дату в формате GMT. Далее, это значение, попадает в сам алгоритма генерации доменных имен, инициалиируя ГСПЧ, который может иметь вид Линейного конгруэтного генератора. Используя одинаковые вектора инициализации, вредоносный файл и владелец ботнета получают идентичные таблицы доменных имен.
        После этого владельцу ботнета достаточно зарегистрировать лишь один домен, для того, чтобы вредоносный файл, рекурсивно посылая запросы к DNS серверу получил IP адрес управляющего сервера для дальнейшей установки с ним соединения и получения, выполнения команд.
        \addimghere{images/dga_work}{1}{Общий принцип работы}{dga_work}
    \subsection{Расмотренные алгоритмы}
    В ходе работы были проанализированы 8 разновидностей алгоритмов генерации доменных имен, а именно:
    \begin{itemize}
        \item Conficker
        \item Cryptolocker
        \item Zeus
        \item PushDo
        \item Rovnix
        \item Tinba
        \item Matsnu
        \item Ramdo
    \end{itemize}
    Для каждого из них, путём обратной разработки были составлены модели работы алгоритмов генерации доменных имен и реализованы на языках программирования высокого уровня. Далее в работе предствлены особенности работы каждого из этих алгоритмов.

Conficker - вирус, впервые появивший в 2088 году, использующий для заражения машин популярную уязвимость MS08-067. Одним из первых применил технику DGA. Процесс генерации доменного имени можно описать пятью шагами, как показано на рис \ref{conficker_dga}  и соответствует самому простому случаю, описанному в главе \ref{work_princip}.
\addimghere{images/conficker}{0.4}{Принцип работы conficker DGA}{conficker_dga}

Cryptolocker - имя вредоносных программ вида троян-вымогатель (ransomware). Целью данного вируса являются системы Microsoft Windows. Cryptolocker польностью шифрует содержимое файловой системы жертвы и требует заплатить выкуп для получения ключа дешифрования.
DGA, который использует cryptolocker относительно прост, однако использует множество приемов, которые осложняют процесс его обратной разработки. Его алгоритм использует для инициализации использует 4 значения - ключ, день, месяц, год.
Ключ может быть констанстой и расчитываться по формуле.
\begin{lstlisting}
Key = (((key * 0x10624DD3) >> 6) * 0xFFFFFC18)+ key)
\end{lstlisting}
В данной работе за значение ключа взята константа 0x41.
Дата, месяц, год инициализируются соответвенно формулам
\begin{lstlisting}
date = ((date<<13 & 0xFFFFFFFF)>>19 & 0xFFFFFFFF) ^ ((date>>1 & 0xFFFFFFFF)<<13 & 0xFFFFFFFF) ^ (date>>19 & 0xFFFFFFFF);
date &= 0xFFFFFFFF;
month = ((month<<2 & 0xFFFFFFFF)>>25 & 0xFFFFFFFF) ^ ((month>>3 & 0xFFFFFFFF)<<7 & 0xFFFFFFFF) ^ (month>>25 & 0xFFFFFFFF);
month &= 0xFFFFFFFF;
year = ((year<<3 & 0xFFFFFFFF)>>11 & 0xFFFFFFFF) ^ ((year>>4 & 0xFFFFFFFF)<<21 & 0xFFFFFFFF) ^ (year>>11 & 0xFFFFFFFF);
year &= 0xFFFFFFFF;
\end{lstlisting}
Каждый символ рассчитывается по формуле
\begin{lstlisting}
chr(ord('a') + (year ^ month ^ date) % 25);
\end{lstlisting}
и его длина составляет
\begin{lstlisting}
date>>3 ^ year>>8 ^ year>>11 & 3 + 12
\end{lstlisting}
В конце добавляется домен верхнего уровня, который последовательно выбирается из массива
\begin{lstlisting}
["com", "net", "biz", "ru", "org", "co.uk", "info"];
\end{lstlisting}
Zeus
PushDo
Rovnix
Tinba
Matsnu
Ramdo

\section{Существующие подходы к классификации}
    \subsection{Naive Bayes}
    \subsection{Logistic Regression}
    \subsection{Random Forest}
    \subsection{Extra Tree Forest}
    \subsection{Voting Classification}
\clearpage

\section{Нейронные сети}
    \subsection{Реккурентные нейронные сети}
    \subsection{LSTM}
    \subsection{Biderectial LSTM}
    \subsection{Механизм внимания}
\clearpage

\section{Эксперимент}
    \subsection{Существующие подходы}
    \subsection{Результаты LSTM}
    \subsection{Сравнительный анализ}
\clearpage