% -*- root: main.tex -*-
\anonsection{Введение}

Некоторые разновидности вредоносных программ используют алгоритмы генерирования доменных имен для определения адресов управляющих серверов. Подобные алгоритмы позволяют защитить вредоносные сервера от однократного отключения или добавления адресов в черные списки. Чаще всего данные алгоритмы используются в крупных ботнетах.

Ботнеты - некоторая сеть, в том числе компьютерная, состоящая из устройств (ботов), со специально запущенным вредоносным программных обеспечением. Чаще всего боты инфицируются посредством вредоносного программного обеспечения, полученного из сети Интернет. Однако путём инфицирования может служить также локальная сеть или устройства ввода, например флэш накопители. Ботнеты являются наиболее распространенным средством кибер атак. Они управляются его создателем при помощи специальных управляющих командных серверов (Command and Control Servers). Большинство из них используются для монетизации различными способами, такими как: распределенные атаки отказ в обслуживании (DDoS атаки), продажа Drive By Download, атаки на клиентов дистанционного банковского обслуживания, для спама и проведения фишинговых атак. Эти и другие угрозы ботсетей выделены в статье [10].

Для удержания контроля над ботами и их управления Ботнеты используют множество способов. Это может быть p2p сети, почтовые протоколы, социальные сети или анонимные сети, такие как TOR или i2p.
Однако самым распространенным на данный момент является Алгоритмы Генерации Доменных Имен (Domain Generation Algotithms).

Они позволяют удерживать контроль над управляющими серверами. В основном подобные алгоритмы используются в крупных ботсетях. Например, одним из первых случаев был компьютерный червь Conficker в 2008 году. На сегодняшний день подобных вредоносных программ насчитываются десятки, каждая из которых представляет серьезную угрозу. Помимо этого, алгоритмы совершенствуются, их обнаружение становится сложнее. Например, осенью 2014 года была обнаружена новая версия ботнета Matsnu, в которой для генерации доменов используются существительные и глаголы из встроенного списка. Подробнее историю развития алгоритмов генерации доменных имен рассмотрена в статье [6].

Целью данной работы является разработка модели на основе методов машинного обучения для распознавания и классификации вредоносных доменных имен, полученных при помощи анализа алгоритмов генерации доменных имен.

Проблемы автоматического анализа алгоритмов DGA и пути их решения можно найти в статье [1]. Идея использования методов машинного обучения освещена в работе [7]. Так, ряд известных компаний, занимающихся информационной безопасностью (Damballa, OpenDns, Click Security и др.), применяют подобные решения для анализа и фильтрации сетевой активности вредоносных программ.
Например, Click Security в своей работе [3] предлагают использовать решающие деревья для бинарной классификации на принадлежность доменов к вредоносным. Для этого ими предложен способ выделения признаков из домена.
Стоит отметить работу [4], которая рассматривает возможность классификации, используя метод опорных векторов (SVM) и выделения из доменов n-gram - подстрока, состоящая из последовательных n символов исходной строки. Схожий подход описывается и в работе [5]. Однако, в отличие от работы [4], имеет большую практическую направленность и предлагает использование алгоритма C4.5.
Подход, основанный на анализе морфем в статье [8], является неактуальным, так как последние исследования[6] показывают, что алгоритмы генерации доменных имен совершенствуются с целью обхода существующих способов обнаружения. Поэтому рассмотрение алгоритмов машинного обучения для предотвращения современных угроз является актуальной проблемой информационной безопасности.

\clearpage