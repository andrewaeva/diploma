% -*- root: main.tex -*-
\anonsection{Введение}

В настоящее время .... вредоносные программы.
Некоторые разновидности вредоносных программ используют алгоритмы генерирования доменных имен для определения адресов управляющих серверов. Подобные алгоритмы позволяют защитить вредоносные сервера от однократного отключения или добавления адресов в черные списки. Чаще всего данные алгоритмы используются в крупных ботнетах. 

Ботнет - некоторая сеть, в том числе компьютерная, состоящая из устройств (ботов), со специально запущенным вредоносным программных обеспечением. Чаще всего боты инфицируются посредством вредоносного программного обеспечения, полученного из сети Интернет. Однако путём инфицирования может служить также локальная сеть или устройства ввода, например флэш накопители. Ботнеты являются наиболее распространенным средством кибер атак. Они управляются его создателем при помощи специальных управляющих командных серверов (Command and Control Servers). Большинство из них используются для монетизации различными способами, такими как: распределенные атаки отказ в обслуживании (DDoS атаки), продажа Drive By Download, атаки на клиентов дистанционного банковского обслуживания, для спама и проведения фишинговых атак. Эти и другие угрозы ботсетей выделены в статье [10].

Для удержания контроля над ботами и их управления Ботнеты используют множество способов. Это может быть одноранговые сети, почтовые протоколы, социальные сети или анонимные сети, такие как TOR или i2p.
Однако самым распространенным на данный момент является Алгоритмы Генерации Доменных Имен (Domain Generation Algotithms).

Они позволяют удерживать контроль над управляющими серверами. В основном подобные алгоритмы используются в крупных ботсетях. Например, одним из первых случаев был компьютерный червь Conficker в 2008 году. На сегодняшний день подобных вредоносных программ насчитываются десятки, каждая из которых представляет серьезную угрозу. Помимо этого, алгоритмы совершенствуются, их обнаружение становится сложнее. Например, осенью 2014 года была обнаружена новая версия ботнета Matsnu, в которой для генерации доменов используются существительные и глаголы из встроенного списка. Подробнее историю развития алгоритмов генерации доменных имен рассмотрена в статье [6].

Целью данной работы является разработка модели на основе методов машинного обучения для распознавания и классификации вредоносных доменных имен, полученных при помощи анализа алгоритмов генерации доменных имен, а также получение количественной оценки качества классификации на сформированной выборке доменных имен.

Для достижения поставленной цели были поставлены три задачи. Во-первых, подготовка эксперементальных данных. При этом необходимо было составить выборки не только для легетимных и вредоносных имен, но и выборки по отдельным классам вредоносных программ. Под вредоносным доменном именем понимается доменное имя, искуственно полученное в результате работы алгоритма генерации доменных имен. Во-вторых, поиск и реализация существующих подходов, их сравнительный анализ на подготовленных данных. В-третьих, разработка и апробация собственной модели на основе реккурентных нейронных сетей.

Именно в третьей задаче и заключается новизна данной работы. Так, для классификации предлагается использование модели, построенной на основе рекуррентной нейронной сети, что является новым направлением исследования для решения проблемы обнаружения и распознования вредоносных доменных имен. Поэтому рассмотрение алгоритмов машинного обучения и нейросетей для предотвращения современных угроз является актуальной проблемой информационной безопасности.

В главе \ref{dga} рассматриваются принципы работы алгоритмов генерации доменных имен.
В главе \ref{be_class} производится обзор существующей лиетературы, затрагивающей данную проблематику. Описываются существующие подходы к классификации, приводятся теоретические основы данных моделей.
В главе \ref{ner_network} рассмотрены теоретические основы и архитектура разработанной модели. Затрагиваются вопросы связанные с нейронными сетями, реккурентными нейронными сетями, моделью Long short-term memory, а также вопросы, связанные с улучшением предложенной модели.
Глава \ref{experiment} посвящена практической реализации существующих подходов \ref{be_class_exp} и разработанной модели \ref{lstm_class_exp}, представлен их сранительный анализ.
\clearpage