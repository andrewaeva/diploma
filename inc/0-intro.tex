% -*- root: main.tex -*-
\anonsection{Введение}
Некоторые разновидности вредоносных программ используют алгоритмы генерирования доменных имен для определения адресов управляющих серверов. Подобные алгоритмы позволяют защитить вредоносные сервера от однократного отключения или добавления адресов в черные списки.

Алгоритмы генерации доменных имен (DGA) часто используются при создании ботсетей [10]. Это позволяет удерживать контроль над управляющими серверами. В основном подобные алгоритмы используются в крупных ботсетях. Например, одним из первых случаев был компьютерный червь Conficker в 2008 году. На сегодняшний день подобных вредоносных программ насчитываются десятки, каждая из которых представляет серьезную угрозу. Помимо этого, алгоритмы совершенствуются, их обнаружение становится сложнее. Например, осенью 2014 года была обнаружена новая версия ботнета Matsnu, в которой для генерации доменов используются существительные и глаголы из встроенного списка.


Целью данной работы является разработка модели на основе алгоритмов машинного обучения для распознавания и классификации вредоносных доменных имен, полученных при помощи алгоритмов генерации доменных имен.

Идея использования методов машинного обучения освещена в работе[7]. Так, ряд известных компаний, занимающихся информационной безопасностью (Damballa, OpenDns, ClickSecurity и др.), применяют подобные решения для анализа и фильтрации сетевой активности вредоносных программ.
Последние исследования [6] показывают, что алгоритмы генерации доменных имен совершенствуются с целью обхода существующих способов обнаружения. Поэтому рассмотрение алгоритмов машинного обучения для предотвращения современных угроз является актуальной проблемой информационной безопасности.




\clearpage
