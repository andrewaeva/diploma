\anonsection{Заключение}
В рамках данной работы автором была разработана модель на основе методов машинного обучения для распознавания и классификации вредоносных доменных имен, полученных при помощи анализа алгоритмов генерации доменных имен, а также получена количественная оценка качества классификации на сформированной выборке доменных имен. Детальный список результатов представляет собой следующий список:
\begin{itemize}
\item Изучены механизмы работы алгоритмов генерации доменных имен;
\item Методом обратной разработки получены и реализованы на языках высокого уровня алгоритмы генерации доменных имен восьми семейств ботнетов;
\item Составлена обучающая выборка, состоящая из легитимных и вредоносных доменных имен;
\item Рассмотрены существующие подходы к распознованию и классификации вредоносных доменных имен;
\item Произведена реализация существующих подходов и получены результаты по каждому из них;
\item Разработана собственная модель, на основе рекуррентной нейронной сети;
\item Произведена реализация и апробация собственной модели распознавания вредоносных доменных имен.
\end{itemize}

В итоге, разработанная модель показывает результат, превосходящий существующие подходы. Данная модель позволяет на тестовой выборке правильно распознать до 97\% вредоносных доменных имен.
Что касается практической применимости, то данная разработка может быть применена и внедрена в такие продукты как: DNS сервера, системы обнаружения вторжений, антивирусные продукты. Это позволит повысить защищенность пользователей от угроз, связанных с распространёнными семействами ботнетов. Такие технологии только выходят на рынок ифнормационной безопасности. Одним из ярких примеров применения похожего подхода является облачный DNS сервис от компании OpenDNS. 

Однако стоит отметить, что процесс распознавания все же остается итерационным, так как постоянное появления всё новых DGA ведет к ухудшению качества модели. Поэтому совершенствование модели должно производиться на всем этапе её жизненого цикла.
\clearpage