\begingroup
\renewcommand{\section}[2]{\anonsection{Библиографический список}}
\begin{thebibliography}{00}
\bibitem{1}
    Н. О. Гончаров.
    Современные угрозы ботсетей. //
    Молодежный научно-технический вестник №10. - октябрь 2014. http://sntbul.bmstu.ru
\bibitem{2}
	T. Barabosch, A. Wichmann, F. Leder, and E. Gerhards-Padilla.
    Automatic extraction of domain name generation algorithms from current malware. //
    Information Assurance and Cyber Defence. - 2012.
\bibitem{3}
    P. Barthakur, M. Dahal, and M. K. Ghose.
    An efficient machine learning based classification scheme for detecting distributed command \& control traffic of p2p botnets. //
    International Journal of Modern Education and Computer Science. - 5(10):9, 2013.
\bibitem{4}
    N. Davuth and S.-R. Kim.
    Classification of malicious domain names using support vector machine and bigram method. //
    International Journal of Security and Its Applications. - 7(1):51–58, January 2013.
\bibitem{5}
    Diederik P. Kingma and Jimmy Ba.
    Adam: A method for stochastic optimization. //
    3rd International Conference for Learning Representations. - San Diego, 2015.
\bibitem{6}
    Junyoung Chung.
    Empirical Evaluation of Gated Recurrent Neural Networks on Sequence Modeling. //
    NIPS 2014 Deep Learning and Representation Learning Workshop. - 2014.
\bibitem{7}
    M. Stevanovic and J. Pedersen.
    Machine learning for identifying botnet network traffic. //
    Technical report, Aalborg Universitet. - 2013.
\bibitem{8}
    Z. Wei-wei, G. Jian, and L. Qian.
    Detecting machine generated domain names based on morpheme features. //
    International Workshop on Clouad Computing and InformationSecurity (CCIS 2013). - Oct. 2013.
\bibitem{9}
    R. J. Williams and D. Zipser. Gradient-Based Learning Algorithms for Recurrent Networks and Their Computational Complexity. In Y. Chauvin and D. E. Rumelhart, editors, Back-propagation: Theory, Architectures and Applications // Lawrence Erlbaum Publishers. - pages 433–486, 1995.
\bibitem{10}
    S. Yadav, A. K. K. Reddy, A. L. N. Reddy, and S. Ranjan.
    Detecting algorithmically generated domain-flux attacks with dns traffic analysis. //
    IEEE/ACM Transactions on Networking  (Volume:20 ,  Issue: 5 ). - 20(5):1663–1677, Oct. 2012.
\bibitem{11}
    Behind the Ramdo DGA. [Электронный ресурс] // Damballa.
    URL: https://www.damballa.com/behind-ramdo-dga-domain-generation-algorithm/
    Режим доступа: свободный, дата обращения: 20.04.2016.
\bibitem{12}
    Christian S. Perone. Machine Learning Text feature extraction (tf-idf). [Электронный ресурс] / Christian S. Perone. // Terra Incognita by Christian S. Perone.
    URL: http://blog.christianperone.com/?p=1589
    Режим доступа: свободный, дата обращения: 25.01.2016.
\bibitem{13}
    Chunting Zhou, Chonglin Sun, Zhiyuan Liu, Francis C.M. Lau.
    A C-LSTM Neural Network for Text Classification. [Электронный ресурс] // arXiv.org.
    URL: http://arxiv.org/abs/1511.08630
    Режим доступа: свободный, дата обращения: 15.04.2016.
\bibitem{14}
    Click Security. Exercise to detect algorithmically generated domain names.
    [Электронный ресурс] // Github.
    URL: https://github.com/ClickSecurity/datahacking/
    Режим доступа: свободный, дата обращения: 12.02.2016.
\bibitem{15}
    Fidelis Threat Advisory \#1016. Pushdo It To Me One More Time.
    [Электронный ресурс] // Threatgeek.
    URL: http://www.threatgeek.com/2015/04/fidelis-threat-advisory-1016-pushdo-it-to-me-one-more-time.html
    Режим доступа: свободный, дата обращения: 20.04.2016.
\bibitem{16}
    J. Jacobs. Building a dga classifier.
    [Электронный ресурс] // Data Driven Security.
    URL: http://datadrivensecurity.info/blog/posts/2014/Oct/dga-part3/, Oct. 2014
    Режим доступа: свободный, дата обращения: 12.02.2016.
\bibitem{17}
    Kyunghyun Cho.
    Learning Phrase Representations using RNN Encoder–Decoder for Statistical Machine Translation. [Электронный ресурс] // arXiv.org.
    URL: https://http://arxiv.org/abs/1406.1078/
    Режим доступа: свободный, дата обращения: 13.04.2016.
\bibitem{18}
    Raff. generation Dgas: A evolution.
    [Электронный ресурс] // Seculert.
    URL: http://www.seculert.com/blog/2014/11/dgas-a-domain-generation-evolution, Nov. 2014.
    Режим доступа: свободный, дата обращения: 12.02.2016.
\bibitem{19}
    Understanding LSTM Networks. [Электронный ресурс] // Colah Blog.
    URL: http://colah.github.io/posts/2015-08-Understanding-LSTMs.
    Режим доступа: свободный, дата обращения: 20.04.2016.
\end{thebibliography}
\endgroup

\clearpage
